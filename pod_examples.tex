\chapter{POD Examples}
\label{ch:pod_examples}

\section{Processing Example #1}
In this example the pod will perform a dynamic orbit determination for PRN04 over a 6 hour arc. The full gravitational force models are applied, with a cannonball model SRP model.

To run the POD ...

$ bin/pod
This should output the following to stdout...

Orbit Determination
Orbit residuals in ICRF : RMS(XYZ)   1.6754034501980351E-002   5.2908718335411935E-002   1.5676115599034774E-002
Orbit Determination: Completed
CPU Time (sec)   298.48134399999998
External Orbit comparison
Orbit comparison: ICRF
RMS RTN   2.8094479714173427E-002   2.4358145601708528E-002   4.4097979280889953E-002
RMS XYZ   1.6754034501980351E-002   5.2908718335411935E-002   1.5676115599034774E-002
Orbit comparison: ITRF
RMS XYZ   3.9069978513805753E-002   3.9343671258381237E-002   1.5660654272651970E-002
Write orbit matrices to output files
CPU Time (sec)   349.19307899999995
The results above show that our orbits arcs, over 6 hours, are currently within 2-5 cm of the final combined IGS orbit.

The processing also produces the following output files...

├── DE.430            planetary ephemris intermediate file
├── Amatrix.out       design matrix
├── Wmatrix.out       reduced observation matrix
├── orbext_ICRF.out   intermediary file for the IGS orbit solution in ICRF for comparison purposes
├── orbext_ITRF.out   intermediary file for the IGS orbit solution in ITRFfor comparison purposes
├── dorb_icrf.out     differences in solutions in ICRF
├── dorb_RTN.out      differences in solutions in orbital frame components radial, tangential and normal (RTN)
├── dorb_Kepler.out   differences in solutions in keperian elements 
├── dorb_itrf.out     differences in solutions in ITRF 
├── orb_icrf.out      the final estimated orbit in ICRF
├── orb_itrf.out      the final estimated orbit in ITRF
├── VEQ_Smatrix.out   State transition matrix from the variational equations solution
├── VEQ_Pmatrix.out   Sensitivity matrix from the variational equations solution


\section{Processing Example #2 - ECOM2 SRP}
In this example we will change the SRP model to use the ECOM2 model.

Edit the EQM.in file so that the Solar Radiation Pressure configuration section now looks:

! Solar Radiation Pressure model: ! 1. Cannonball model ! 2. Box-wing model ! 3. ECOM (D2B1) model SRP\_model 3

Then edit VEQ.in, so that the Non-gravitational forces now looks like:

%% Non-gravitational Effects Solar_radiation 0 Earth_radiation 0 Antenna_thrust 0

! Solar Radiation Pressure model: ! 1. Cannonball model ! 2. Box-wing model ! 3. ECOM (D2B1) model SRP\_model 3

run the POD ...

$ bin/pod
This should output the following to stdout...

Orbit Determination
Orbit residuals in ICRF : RMS(XYZ)   2.0336204859568077E-002   8.4715644601919167E-003   3.9687932322714677E-002
Orbit Determination: Completed
CPU Time (sec)   299.68054799999999
External Orbit comparison
Orbit comparison: ICRF
RMS RTN   2.8182836396022540E-002   2.4598832384842121E-002   2.5879201921952168E-002
RMS XYZ   2.0336204859568077E-002   8.4715644601919167E-003   3.9687932322714677E-002
Orbit comparison: ITRF
RMS XYZ   1.8757217704973204E-002   1.1635302426688266E-002   3.9702619816620370E-002
Write orbit matrices to output files
CPU Time (sec)   350.88653299999999\


\section{Example 3 - (pod/examples/ex3)}
GPS IGS SP3 file orbit fitting, orbit prediction and comparison to next IGS SP3 file

\section{Example 4 - (pod/examples/ex4):}
Integration of POD initial conditions file generated by the PEA

\section{Example 5 - (pod/examples/ex5):}
ECOM1+ECOM2 hybrid SRP model
In each example directory (ex1/ex2/ex3/ex4) there is a sh\_ex? script that when executed will run the example and compare the output with the expected solution.

\newthought{The POD} is designed to do some stuff.