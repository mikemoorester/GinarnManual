% This is a modified version of the tufte-latex book example in which the title page and the contents page resemble Tufte's VDQI book, using Kevin Godby's code from this thread at https://groups.google.com/forum/#!topic/tufte-latex/ujdzrktC1BQ.
%
%%%%%%%%%%%%%%%%%%%%%%%%%%%%%%%%%%%%%%%%%%%%%%%%%%%%%%%%%%%%%%%%%%%%%%
% How to use Overleaf: 
%
% You edit the source code here on the left, and the preview on the
% right shows you the result within a few seconds.
%
% Bookmark this page and share the URL with your co-authors. They can
% edit at the same time!
%
% You can upload figures, bibliographies, custom classes and
% styles using the files menu.
%
% If you're new to LaTeX, the wikibook is a great place to start:
% http://en.wikibooks.org/wiki/LaTeX
%
%%%%%%%%%%%%%%%%%%%%%%%%%%%%%%%%%%%%%%%%%%%%%%%%%%%%%%%%%%%%%%%%%%%%%%
%% Unfortunately for the contents to contain
%% the "Parts" lines successfully, hyperref
%% needs to be disabled.
\documentclass[nohyper,nobib]{tufte-book}
\usepackage{nameref}
% \hypersetup{colorlinks}% uncomment this line if you prefer colored hyperlinks (e.g., for onscreen viewing)

% \usepackage{hyphenat}
\usepackage{url}
\usepackage[backend=biber, natbib=true, style=numeric]{biblatex}
\addbibresource{sample-handout.bib}
\usepackage{xargs}
\renewcommandx{\cite}[3][1={0pt},2={}]{\sidenote[][#1]{\fullcite[#2]{#3}}}

%%
% Book metadata
\title{Ginarn}
\date{Version 0.0}
\author[The Ginarn team]{
Sebastien Allegyer,
Rupert Brown,
John Donovan,
Ken Harima,
Aaron Hammond,
Lavanya Kumarappan,
Tao Li,
Ronald Maj,
Bogdan Matviichuk,
Chris Morgan,
Michael Moore,
Simon McClusky,
Thomas Papanikolaou,
Vincent Rooke,
Tzupang Tseng
}
\publisher{Geoscience Australia and Frontier-SI}

%%
% If they're installed, use Bergamo and Chantilly from www.fontsite.com.
% They're clones of Bembo and Gill Sans, respectively.
%\IfFileExists{bergamo.sty}{\usepackage[osf]{bergamo}}{}% Bembo
%\IfFileExists{chantill.sty}{\usepackage{chantill}}{}% Gill Sans

%\usepackage{microtype}

%%
% Just some sample text
\usepackage{lipsum}

%%
% For nicely typeset tabular material
\usepackage{booktabs}

%%
% For graphics / images
\usepackage{graphicx}
\setkeys{Gin}{width=\linewidth,totalheight=\textheight,keepaspectratio}
\graphicspath{{graphics/}}

% The fancyvrb package lets us customize the formatting of verbatim
% environments.  We use a slightly smaller font.
\usepackage{fancyvrb}
\fvset{fontsize=\normalsize}

\usepackage{tikz}
%\usetikzlibrary{external}
%\tikzexternalize[prefix=tikz/]
\usetikzlibrary{shapes.geometric, arrows}
%%
% Prints argument within hanging parentheses (i.e., parentheses that take
% up no horizontal space).  Useful in tabular environments.
\newcommand{\hangp}[1]{\makebox[0pt][r]{(}#1\makebox[0pt][l]{)}}

%%
% Prints an asterisk that takes up no horizontal space.
% Useful in tabular environments.
\newcommand{\hangstar}{\makebox[0pt][l]{*}}

%%
% Prints a trailing space in a smart way.
\usepackage{xspace}

\usepackage{amsmath}
%%
% Some shortcuts for Tufte's book titles.  The lowercase commands will
% produce the initials of the book title in italics.  The all-caps commands
% will print out the full title of the book in italics.
\newcommand{\vdqi}{\textit{VDQI}\xspace}
\newcommand{\ei}{\textit{EI}\xspace}
\newcommand{\ve}{\textit{VE}\xspace}
\newcommand{\be}{\textit{BE}\xspace}
\newcommand{\VDQI}{\textit{The Visual Display of Quantitative Information}\xspace}
\newcommand{\EI}{\textit{Envisioning Information}\xspace}
\newcommand{\VE}{\textit{Visual Explanations}\xspace}
\newcommand{\BE}{\textit{Beautiful Evidence}\xspace}

\newcommand{\TL}{Tufte-\LaTeX\xspace}

% Prints the month name (e.g., January) and the year (e.g., 2008)
\newcommand{\monthyear}{%
  \ifcase\month\or January\or February\or March\or April\or May\or June\or
  July\or August\or September\or October\or November\or
  December\fi\space\number\year
}


% Prints an epigraph and speaker in sans serif, all-caps type.
\newcommand{\openepigraph}[2]{%
  %\sffamily\fontsize{14}{16}\selectfont
  \begin{fullwidth}
  \sffamily\large
  \begin{doublespace}
  \noindent\allcaps{#1}\\% epigraph
  \noindent\allcaps{#2}% author
  \end{doublespace}
  \end{fullwidth}
}

% Inserts a blank page
\newcommand{\blankpage}{\newpage\hbox{}\thispagestyle{empty}\newpage}

\usepackage{units}

% Typesets the font size, leading, and measure in the form of 10/12x26 pc.
\newcommand{\measure}[3]{#1/#2$\times$\unit[#3]{pc}}

% Macros for typesetting the documentation
\newcommand{\hlred}[1]{\textcolor{Maroon}{#1}}% prints in red
\newcommand{\hangleft}[1]{\makebox[0pt][r]{#1}}
\newcommand{\hairsp}{\hspace{1pt}}% hair space
\newcommand{\hquad}{\hskip0.5em\relax}% half quad space
\newcommand{\TODO}{\textcolor{red}{\bf TODO!}\xspace}
\newcommand{\ie}{\textit{i.\hairsp{}e.}\xspace}
\newcommand{\eg}{\textit{e.\hairsp{}g.}\xspace}
\newcommand{\na}{\quad--}% used in tables for N/A cells
\providecommand{\XeLaTeX}{X\lower.5ex\hbox{\kern-0.15em\reflectbox{E}}\kern-0.1em\LaTeX}
\newcommand{\tXeLaTeX}{\XeLaTeX\index{XeLaTeX@\protect\XeLaTeX}}
% \index{\texttt{\textbackslash xyz}@\hangleft{\texttt{\textbackslash}}\texttt{xyz}}
\newcommand{\tuftebs}{\symbol{'134}}% a backslash in tt type in OT1/T1
\newcommand{\doccmdnoindex}[2][]{\texttt{\tuftebs#2}}% command name -- adds backslash automatically (and doesn't add cmd to the index)
\newcommand{\doccmddef}[2][]{%
  \hlred{\texttt{\tuftebs#2}}\label{cmd:#2}%
  \ifthenelse{\isempty{#1}}%
    {% add the command to the index
      \index{#2 command@\protect\hangleft{\texttt{\tuftebs}}\texttt{#2}}% command name
    }%
    {% add the command and package to the index
      \index{#2 command@\protect\hangleft{\texttt{\tuftebs}}\texttt{#2} (\texttt{#1} package)}% command name
      \index{#1 package@\texttt{#1} package}\index{packages!#1@\texttt{#1}}% package name
    }%
}% command name -- adds backslash automatically
\newcommand{\doccmd}[2][]{%
  \texttt{\tuftebs#2}%
  \ifthenelse{\isempty{#1}}%
    {% add the command to the index
      \index{#2 command@\protect\hangleft{\texttt{\tuftebs}}\texttt{#2}}% command name
    }%
    {% add the command and package to the index
      \index{#2 command@\protect\hangleft{\texttt{\tuftebs}}\texttt{#2} (\texttt{#1} package)}% command name
      \index{#1 package@\texttt{#1} package}\index{packages!#1@\texttt{#1}}% package name
    }%
}% command name -- adds backslash automatically
\newcommand{\docopt}[1]{\ensuremath{\langle}\textrm{\textit{#1}}\ensuremath{\rangle}}% optional command argument
\newcommand{\docarg}[1]{\textrm{\textit{#1}}}% (required) command argument
\newenvironment{docspec}{\begin{quotation}\ttfamily\parskip0pt\parindent0pt\ignorespaces}{\end{quotation}}% command specification environment
\newcommand{\docenv}[1]{\texttt{#1}\index{#1 environment@\texttt{#1} environment}\index{environments!#1@\texttt{#1}}}% environment name
\newcommand{\docenvdef}[1]{\hlred{\texttt{#1}}\label{env:#1}\index{#1 environment@\texttt{#1} environment}\index{environments!#1@\texttt{#1}}}% environment name
\newcommand{\docpkg}[1]{\texttt{#1}\index{#1 package@\texttt{#1} package}\index{packages!#1@\texttt{#1}}}% package name
\newcommand{\doccls}[1]{\texttt{#1}}% document class name
\newcommand{\docclsopt}[1]{\texttt{#1}\index{#1 class option@\texttt{#1} class option}\index{class options!#1@\texttt{#1}}}% document class option name
\newcommand{\docclsoptdef}[1]{\hlred{\texttt{#1}}\label{clsopt:#1}\index{#1 class option@\texttt{#1} class option}\index{class options!#1@\texttt{#1}}}% document class option name defined
\newcommand{\docmsg}[2]{\bigskip\begin{fullwidth}\noindent\ttfamily#1\end{fullwidth}\medskip\par\noindent#2}
\newcommand{\docfilehook}[2]{\texttt{#1}\index{file hooks!#2}\index{#1@\texttt{#1}}}
\newcommand{\doccounter}[1]{\texttt{#1}\index{#1 counter@\texttt{#1} counter}}

% Generates the index
\usepackage{makeidx}
\makeindex

%%%% Kevin Godny's code for title page and contents from https://groups.google.com/forum/#!topic/tufte-latex/ujdzrktC1BQ
\makeatletter
\renewcommand{\maketitlepage}{%
\begingroup%
\setlength{\parindent}{0pt}

{\fontsize{24}{24}\selectfont\textit{\@author}\par}

\vspace{1.75in}{\fontsize{36}{54}\selectfont\@title\par}

\vspace{0.5in}{\fontsize{14}{14}\selectfont\textsf{\smallcaps{\@date}}\par}

\vfill{\fontsize{14}{14}\selectfont\textit{\@publisher}\par}

\thispagestyle{empty}
\endgroup
}
\makeatother

\titlecontents{part}%
    [0pt]% distance from left margin
    {\addvspace{0.25\baselineskip}}% above (global formatting of entry)
    {\allcaps{Part~\thecontentslabel}\allcaps}% before w/ label (label = ``Part I'')
    {\allcaps{Part~\thecontentslabel}\allcaps}% before w/o label
    {}% filler and page (leaders and page num)
    [\vspace*{0.5\baselineskip}]% after

\titlecontents{chapter}%
    [4em]% distance from left margin
    {}% above (global formatting of entry)
    {\contentslabel{2em}\textit}% before w/ label (label = ``Chapter 1'')
    {\hspace{0em}\textit}% before w/o label
    {\qquad\thecontentspage}% filler and page (leaders and page num)
    [\vspace*{0.5\baselineskip}]% after
%%%% End additional code by Kevin Godby

\begin{document}

% Front matter
\frontmatter

% v.2 epigraphs
% \newpage\thispagestyle{empty}
% \openepigraph{%
% The public is more familiar with bad design than good design.
% It is, in effect, conditioned to prefer bad design, 
% because that is what it lives with. 
% The new becomes threatening, the old reassuring.
% }{Paul Rand%, {\itshape Design, Form, and Chaos}
% }
% \vfill
% \openepigraph{%
% A designer knows that he has achieved perfection 
% not when there is nothing left to add, 
% but when there is nothing left to take away.
% }{Antoine de Saint-Exup\'{e}ry}
% \vfill
% \openepigraph{%
% \ldots the designer of a new system must not only be the implementor and the first 
% large-scale user; the designer should also write the first user manual\ldots 
% If I had not participated fully in all these activities, 
% literally hundreds of improvements would never have been made, 
% because I would never have thought of them or perceived 
% why they were important.
% }{Donald E. Knuth}


% r.3 full title page
\maketitle


% v.4 copyright page
\newpage
\begin{fullwidth}
~\vfill
\thispagestyle{empty}
\setlength{\parindent}{0pt}
\setlength{\parskip}{\baselineskip}
Copyright \copyright\ \the\year\ \thanklessauthor

\par\smallcaps{Published by \thanklesspublisher}

\par\smallcaps{tufte-latex.googlecode.com}

\par Licensed under the Apache License, Version 2.0 (the ``License''); you may not
use this file except in compliance with the License. You may obtain a copy
of the License at \url{http://www.apache.org/licenses/LICENSE-2.0}. Unless
required by applicable law or agreed to in writing, software distributed
under the License is distributed on an \smallcaps{``AS IS'' BASIS, WITHOUT
WARRANTIES OR CONDITIONS OF ANY KIND}, either express or implied. See the
License for the specific language governing permissions and limitations
under the License.\index{license}

\par\textit{First printing, \monthyear}
\end{fullwidth}
% r.5 contents
\tableofcontents
%
\include{welcome}
\newthought{This manual} is divided into three major sections: the preliminary matter which gives a quick overview on how to install and use the software with examples. The next part contains the background theory on the models that have been implemented into Ginarn. The back matter contains information about configuration files, and file formats used by the software.
%
\listoffigures
%
\listoftables
%
% r.9 introduction
\cleardoublepage
\part{Overview}
%
\include{introduction}
%
\chapter{Installation}
\label{ch:installation}


\section{To Install} 

In this section we will describe how to install the PEA and POD from source.

\subsection{PEA}


\newthought{Dependencies} the following packages need to be installed with the minimum versions as shown below. This guide will outline the preferred method of installation.

CMAKE  > 3.0 requires openssl-devel to be installed (requires openssl-devel)
YAML   > 0.6
Boost  > 1.70
gcc    > 4.1
Eigen3
Build
To build the PEA Precise Estimation Algorithm...

We suggest using the following directory structure when installing the ACS toolkit. It will be created by following this guide.

%/data/
%└── acs/
%    ├── pea/
%    └── pod/

The following is an example procedure to install the dependencies necessary to run the pea on a base ubuntu linux distribution

Update the base operating system:
\begin{verbatim}
$ sudo apt update
$ sudo apt upgrade
\end{verbatim}
Install base utilities gcc, gfortran, git, openssl, blas, lapack, etc
\begin{verbatim}
$ sudo apt install -y git gobjc gobjc++ gfortran libopenblas-dev openssl curl net-tools openssh-server cmake make \
liblapack-dev gzip vim libssl1.0-dev python3-cartopy python3-scipy python3-matplotlib python3-mpltoolkits.basemap
Create a temporary directory structure to make the dependencies in:
$ sudo mkdir -p /data/tmp
$ cd /data/tmp
\end{verbatim}

\newthought{YAML}
We are using the YAML library to parse the configuration files used to run many of the programs found in this library (https://github.com/jbeder/yaml-cpp). Here is an example of how we have installed the yaml library from source:
\begin{verbatim}
$ cd /data/tmp
$ sudo git clone https://github.com/jbeder/yaml-cpp.git
$ cd yaml-cpp
$ sudo mkdir cmake-build
$ cd cmake-build
$ sudo cmake .. -DCMAKE\_INSTALL\_PREFIX=/usr/local/ -DYAML\_CPP\_BUILD\_TESTS=OFF
$ sudo make install yaml-cpp
$ cd ../..
$ sudo rm -fr yaml-cpp
\end{verbatim}

\newthought{Boost}
We rely on a number of the utilities provided by boost (https://www.boost.org/), such as their time and logging libraries.
\begin{verbatim}

$ cd /data/tmp/
$ sudo wget -c https://dl.bintray.com/boostorg/release/1.73.0/source/boost_1_73_0.tar.gz
$ sudo gunzip boost_1_73_0.tar.gz
$ sudo tar xvf boost_1_73_0.tar
$ cd boost_1_73_0/
$ sudo ./bootstrap.sh
$ sudo ./b2 install
$ cd ..
$ sudo rm -fr boost_1_73_0/ boost_1_73_0.tar

\end{verbatim}

\newthought{Eigen3} is used for performing matrix calculations, and has a very nice API.
\begin{verbatim}
$ cd /data/tmp/
$ sudo git clone https://gitlab.com/libeigen/eigen.git
$ cd eigen
$ sudo mkdir cmake-build
$ cd cmake-build
$ sudo cmake ..
$ sudo make install
$ cd ../..
$ sudo rm -rf eigen
Installing PEA
PEA Executable
$ cd /data/acs/
\end{verbatim}
Clone the repository via https:

\begin{verbatim}
$ git clone https://bitbucket.org/geoscienceaustralia/pea.git
\end{verbatim}
You should now have...

%pea
%├── INSTALL.md
%├── LICENSE.md
%├── README.md
%├── aws/                - for automated builds in aws
%├── config/
%│   ├── Ex00-UnitTest.yaml
%│   ├── Ex01-PPP.yaml
%│   ├── Ex02-Network.yaml
%│   ├── Ex03-Network_Orbits.yaml
%│   ├── Ex04-Ionosphere.yaml
%│   ├── Ex05-Realtime.yaml
%│   ├── iontest_20115w.yaml
%│   └── PPP-iontest.yaml
%├── cpp/
%│   ├── CMakeLists.txt
%│   ├── cmake           - files to help cmake find dependencies
%│   ├── docs            - automatic code documentation configuration
%│   └── src/
%│       ├── 3rdparty/   - see ACKNOWLEDGEMENTS in README.md
%│       ├── common/     - libraries used by the pea
%│       ├── iono/       - routines for ionosphere modelling
%│       ├── pea/        - main for `pea`
%│       └── rtklib/     - subset of modified routines from RTKlib see ACKNOWLEDGEMENTS in README.md
%└── python
%    ├── config
%    ├── README.md
%    └── source
%        ├── download_examples.py
%        ├── install_examples.py
%        └── other helper programs
Prepare a directory to build in, its better practise to keep this separated from the source code.
\begin{verbatim}
$ cd pea/cpp
$ mkdir -p build
$ cd build
\end{verbatim}

Run cmake to find the build dependencies and create the make file:

\begin{verbatim}
$ cmake ..
\end{verbatim}$ 
Now build the pea

\begin{verbatim}
$ cmake --build $PWD --target pea
\end{verbatim}

To change to build location substitute your preferred destination for $PWD , e.g /usr/local/bin

Alternatively to the command above you can make the code in parallel using:
\begin{verbatim}
$ make -j 5 all
\end{verbatim}

where the -j flag controls how many jobs can be run at the same time.

Check to see if you can execute the pea:
\begin{verbatim}
$ ./pea    
\end{verbatim}

and you should see something similar to:
\begin{verbatim}
PEA starting...
Options:
  --help                      Help
  --verbose                   More output
  --quiet                     Less output
  --config arg                Configuration file
  --trace_level arg           Trace level
  --antenna arg               ANTEX file
  --navigation arg            Navigation file
  --sinex arg                 SINEX file
  --sp3file arg               Orbit (SP3) file
  --clkfile arg               Clock (CLK) file
  --dcbfile arg               Code Bias (DCB) file
  --ionfile arg               Ionosphere (IONEX) file
  --podfile arg               Orbits (POD) file
  --blqfile arg               BLQ (Ocean loading) file
  --erpfile arg               ERP file
  --elevation_mask arg        Elevation Mask
  --max_epochs arg            Maximum Epochs
  --epoch_interval arg        Epoch Interval
  --rnx arg                   RINEX station file
  --root_input_dir arg        Directory containg the input data
  --root_output_directory arg Output directory
  --start_epoch arg           Start date/time
  --end_epoch arg             Stop date/time
  --dump-config-only          Dump the configuration and exit
PEA finished
\end{verbatim}

\newthought{The documentation} for the pea can be generated similarly using doxygen if it is installed.

\begin{verbatim}
$ sudo apt-get install doxygen
$ cd pea/cpp/build
$ make doc_doxygen
\end{verbatim}
The docs can then be found at doc\_doxygen/html/index.html

\subsection{POD from source}

%The ACS Version 0.0.1 beta release supports:

%The POD
%Directory Structure
%pod/
%├── LICENSE.md
%├── INSTALL.md
%├── README.md
%├── src/
%├── bin/  (created)
%├── lib/  (created)
%├── config/
%├── tables/
%├── scripts/

Dependencies

The open basic linear algebra library (Openblas.x86\_64,liblas-libs.x86\_64) (You may need to run the command ln -s /usr/lib64/libopenblas.so.3 /usr/lib64/libopenblas.so)
A working C compiler (gcc will do), a working C++ compiler (gcc-g++ will do) and a fortran compiler (we have used gfortran)
Cmake (from cmake.org) at least version 2.8
If the flags set in CMakeLists.txt do not work with your compiler please remove/replace the ones that don't

Build
To build the POD ...
\begin{verbatim}

$ cd pod
$ mkdir build
$ cd build
$ cmake3 .. 
$ make >make.out 2>make.err
$ less make.err (to verify everything was built correctly)
    
\end{verbatim}
You should now have the executables in the bin directory: pod crs2trs brdc2ecef

Test
To test your build of the POD ... - You may not need the ulimit command but we found it necessary

\begin{verbatim}

$ cd ../pod/test
$ ulimit -s unlimited
$ ./sh_test_pod
At the completion of the test run, the sh_test_pod script will return any differences to the standard test resuts
    
\end{verbatim}

Configuration File
The POD Precise Orbit Determination (./bin/pod) uses the configuration file: 
%├── EQM.in (Full force model equation of motion) ├── VEQ.in (For variational equations) ├── POD.in (For all other config)


%\section{From precompiled binaries}
%
%To do..
%
%\include{implementation overview of software}
% block diagram from aaron
%
\chapter{PEA examples}
\label{ch:pea_examples}


\section{Example 01 - Use PPP to Process GPS Data}
In this example we will process 24 hours of data from a permanent reference frame station. The algorithm that will be use an L1+L2 and L1+L5 ionosphere free combination. The log files and processing results can be found in /data/acs/pea/output/exs/EX01_IF/.

$ ./pea --config ../../config/Ex01-IF-PPP.yaml

$ grep "\$POS" /data/acs/pea/output/exs/EX01_IF/EX01_IF-ALIC.TRACE
And you should see the following: <snip> $POS,2062,431940.000,0,-4052052.7956,4212836.0144,-2545104.6423,0.00000043966020,0.00000039738502,0.00000013421476 $POS,2062,431970.000,0,-4052052.7956,4212836.0144,-2545104.6423,0.00000043965772,0.00000039738393,0.00000013421667

For more information on running the pea in ppp mode see docs/PPPExamples.md

\section{Example 02 Processing a Global Network to obtain satellite clock products}
In this example 24 hours of data from a small global network of 87 stations is processed to obtain the clock products needed for high precision positioning.

Check that the paths in the configuration file for the products and RINEX files are correct for your system. If you have followed the convention layed out in the INSTALL.md document you should not need to amend anything.

To start the processing use the command:

$ ./pea --config ../../config/Ex02-Network.yaml
The process will take approximatelly 2-3 hours to complete depending on CPU performance. The log files and processing results can be found in /data/acs/pea/output/examples/Ex02, or the alternative directory you have specified in the configuration file.

Change into your output directory. You should find a .TRACE file for each station processed, and a PEA.SUM file.

$ cd /data/acs/pea/output/examples/Ex02
To verify your solution, first grep for the xp values:

$ grep 'network xp' netprocessing.out > xp_test.txt
and then run:

$ python3 /data/acs/pea/python/src/comprun.py --test /data/acs/output/xp_test.txt --standard /data/acs/pea/example/EX02/standard/xp_standard.txt
This will produce the plots Posdiff.png, recclk.png, satclk.png, and zwd.png.

\section{Example 03 Processing a Global Network to obtain the orbit and clock products}
In this example 24 hours of data from a small global network of 87 stations is processed to obtain the orbit and clock products needed for high precision positioning.

Check that the paths in the configuration file for the products and RINEX files are correct for your system. If you have followed the convention layed out in the INSTALL.md document you should not need to amend anything.

To start the processing use the command:

$ ./pea --config ../../config/Ex03-Network_Orbits.yaml
The process will take approximatelly 2-3 hours to complete depending on CPU performance. The log files and processing results can be found in /data/acs/pea/output/examples/Ex03, or the alternative directory you have specified in the configuration file.

Change into your output directory. You should find a .TRACE file for each station processed, and a PEA.SUM file.

$ cd /data/acs/pea/output/examples/Ex03
To verify your solution, first grep for the xp values:

$ grep 'network xp' netprocessing.out > xp_test.txt
and then run:

$ python3 /data/acs/pea/python/src/comprun.py --test /data/acs/output/xp_test --standard /data/acs/pea/example/EX03/standard/xp_standard.txt
This will produce the plots Posdiff.png, recclk.png, satclk.png, and zwd.png.

To compare the satellite clocks run:

$ python3 /data/acs/pea/python/src/compareclk.py --standard /data/acs/pea/example/EX03/standard/aus20624.clk  --test ./aus20624.clk
This will produce plots for the differences in satellite clocks G02 through to G32 as well as calculating the RMS and standard deviation with respect to the standard. G01.png does not exists at this has been used as the pivot satellite clock that we use to remove the bias from.

\section{Example 04 Processing a Global Network to obtain Ionospheric Vertical Total Electron Content (VTEC) Maps}
In this example 24 hours of data from a small global network of 87 stations is processed to obtain the IONEX formatted Ionosphere VTEC maps and SINEX formatted satellite Diferential Signal Biases (DSB). Ionospheric VTEC maps follows the IONEX 1.1 format, which can be found in: https://gssc.esa.int/wp-content/uploads/2018/07/ionex11.pdf The satellite bias follows the bias-SINEX format, which can be found in: http://ftp.aiub.unibe.ch/bcwg/format/draft/sinex_bias_100_dec07.pdf

Check that the paths in the configuration file for the products and RINEX files are correct for your system. If you have followed the convention layed out in the INSTALL.md document you should not need to amend anything.

To start the processing use the command:

$ ./pea --config <installation directory>/pea/config/Ex04-Ionosphere.yaml
The process will take approximatelly 1-2 hours to complete depending on CPU performance and setting specifications.

The IONEX and bias-SINEX files can then be used to obtain SPP and PPP positioning solutions as specified in PPPExamples.md

A utility to plot the TEC values in an IONEX file have can be found at:

<installation dirctory>/pea/python/source/plotIONEX.py
This utility will ask for the path/filename of the IONEX file and a start and stop time in hhmmss format. The Python code will then generate a figure in IONEX_yyyy-mm-dd_hh:mm:ss.png format for each entry between the start and stop time.

\section{Example 05 Real-time}

\newthought{The PEA} is designed to do some stuff.
%
\include{pod_examples}
%
% Start the main matter (normal chapters)
\mainmatter

\part{Background Theory}
\include{GNSSOverview}
%
\include{observation_modelling}
%
\include{kalman_filter}
%
\chapter{Orbit Modelling}
\label{ch:orbit_modelling}


\newthought{POD} is designed to do some stuff.

\section{Gravitational Force Models}

\section{Non-Gravitational Force Models}

\subsection{Solar Radiation Force Models}

\subsection{Cannonball}

\subsection{ECOM I}

\subsection{ECOM II}

\subsection{ECOM C}

\subsection{Box Wing}

\subsection{Antenna Thrust}

\subsection{Albedo}

\section{Transformtation between Celestial and Terrestrial Reference Systems}

The variational equations obtained from the POD need to be transformed into the terrestrial reference frame so that the adjustments can be made in the ECEF frame that the PEA operates in.

\begin{equation}
    [CRS] = Q(t)R(t)W(t)[TRS]
\end{equation}

Where,
$CRS$ is the Celestial Reference System
$TRS$ is the Terrestrial Reference Systems
$Q(t)$ is the Celestial Pole motion (Precession-Nutation) matrix
$R(t)$ is the Earth Rotation matrix
$W(t)$ is the Polar motion matrix
\\
\begin{equation}
Q(t) = 
\begin{bmatrix} 
1-aX^2  & -aXY     & X \\
 -aXY   & 1 - aY^2 & Y \\
 -X     & -Y       & 1-a(X^2+Y^2) 
\end{bmatrix}
\end{equation}
\\
\begin{equation}
R(t) = R_2(-\theta) = 
\begin{bmatrix}
cos \theta & -sin \theta & 0 \\
sin \Theta & cos \theta  & 0 \\
0 & 0 1
\end{bmatrix}
\end{equation}
\\
\begin{equation}
    W(t) = R_z (-s^') R_y(x_p) R_x(y_p)
\end{equation}




%
\include{ionosphere}
%
\include{ambiguity_resolution}
%
%\include{styleguide}
%%
% The back matter contains appendices, bibliographies, indices, glossaries, etc.
\backmatter
%\include{styleguide}
% \bibliography{sample-handout}
% \bibliographystyle{plainnat}
\printindex

\end{document}
